
\documentclass[runningheads]{llncs}

\usepackage[utf8]{inputenc}
\usepackage{threeparttable}
\usepackage{natbib}
\usepackage{amsmath,bm}
\usepackage{amssymb}
\setcounter{tocdepth}{3}
\usepackage{graphicx}
\usepackage{subcaption}
\captionsetup{compatibility=false}
\DeclareGraphicsExtensions{.eps,.pdf,.png,.jpg,.jpeg,.xcf}
\graphicspath{{./figures/}} % Specifies the directory where pictures are stored
\usepackage{epstopdf, epsfig}
\epstopdfsetup{outdir=./figures/}
\usepackage[english]{babel}
\usepackage{color,soul} %highlight with color
\newcommand{\changes}[1]{\textcolor{magenta}{#1}}     %%defined a new function to trace changes by modify font color
\newcommand{\todo}[1]{\textcolor{red}{#1}} 

%-----------------text wrap in table-----------------------------
\usepackage{array}
\newcolumntype{L}{>{\centering\arraybackslash}m{4cm}}
%-----------------------------------------------------------------
\begin{document}

\mainmatter 

\title{ \Large Anisotropic Thermal Conduction proxy-app in Nektar++ framework:\\
	\vspace{0.5cm}
\large Derivation of variational formulation}
\titlerunning{anisotropic thermal conduction}

\author{}
\institute{}

\authorrunning{}


\toctitle{Abstract}
\tocauthor{{}}

\maketitle


\begin{abstract}
A tokamak is currently the leading candidate for a practical fusion rector using the magnetic confinement approach to produce controlled thermonuclear fusion power. The transport of heat in magnetized plasma is important to the stability and control of thermonuclear fusion process in tokamak. Due to the presence of magnetic field and the ionized nature of plasma, the thermal conduction of the magnetized plasma is anisotropic. Due to the anisotropy of thermal conduction, the diffusion tensor becomes a non-diagonal matrix and a sense of directionality is introduced into the normally uniformly dissipative characteristics of the diffusion term. Nektar++~\cite{cantwell2015nektar++} is a spectral/hp element framework that is designed to simplify the task of applying high-order finite element methods in a variety of applications. In this project, the variational formulation the anisotropic thermal conduction in the magnetic plasma is derived and implemented in Nektar++ framework, and the key parameters are briefly explained.   
\end{abstract}

\medskip

\begingroup
\let\clearpage\relax
\tableofcontents
\endgroup

\medskip
\medskip

\section{Governing equation} \label{sec:geq}
The description of plasma physics mainly include three ways: single particle approach, kinetic theory and fluid theory~\cite{dendy1995plasma}. One of the most popular approaches that coarse-grain microscopic plasma kinetics is the fluid approach. In a magnetized plasma, this description of the plasma is called magnetohydrodynamics (MHD) model. The derivation of the MHD model starts from the collisionless Boltzmann equation in Equation~\eqref{eq:plasma1} which is the key equation for all kinetic theory including plasma, where $f$ is a function of the spatial, velocity and time spaces and represents the number of particles in at position $\bm{x}$, particle velocity space $\bm{v}$ and time $t$. The $(\bm{x},\bm{v})$ is called the phase space.
\begin{eqnarray}
\frac{\partial f}{\partial t} + \bm{v}\cdot \frac{\partial f}{\partial \bm{x}}+\bm{a}\cdot\frac{\partial f}{\partial \bm{v}} = 0 \label{eq:plasma1}
\end{eqnarray}
The $\bm{v}$ and $\bm{a}$ are the velocity and acceleration of a particle. To describe the plasma physics, the acceleration $(\bm{F}/m = \bm{a})$ is chosen to be the Lorentz force in Equation~\eqref{eq:plasma2}
\begin{eqnarray}
\bm{F} = m\bm{a} = q(\bm{E}+\bm{v}\times\bm{B}) \label{eq:plasma2}
\end{eqnarray}
where $m$, $q$, $\bm{E}$ and $\bm{B}$ respectively are the mass of particle, the electrical charge, the strength of electrical and magnetic fields. The resultant equation is called the Vlasov equation, as shown in Equation~\eqref{eq:plasma3}, 
\begin{eqnarray}
\frac{\partial f}{\partial t} + \bm{v}\cdot \frac{\partial f}{\partial \bm{x}} + \frac{q[\bm{E}+\bm{v}\times \bm{B}]}{m}\cdot \frac{\partial f}{\partial \bm{v}} = 0\label{eq:plasma3}
\end{eqnarray} 

In plasma, the interaction of particles is via Coulomb forces, which is over a long range and weak, compared with direct collisions of particles. If the plasma is only partially ionized, collisions with neutrals will occur~\cite{dendy1995plasma}. On the other hand, in a fully ionized plasma, as well as the cooperative effects described by the fields $\bm{E}$ and $\bm{B}$, particles do experience micro-collisions, e.g., they are gradually deflected by large number of small deflections caused by local Coulomb interactions. This is also the cause of the ultimate thermalization. To include the collision effect, a collision operator $\sum_j C(f_i, f_j)$ is inserted into the Vlasov equation as an external force. Hence Equation~\eqref{eq:plasma3} becomes the form in Equation~\eqref{eq:plasma4}, collisional Boltzmann-Vlasov equation. 
\begin{eqnarray}
\frac{\partial f}{\partial t} + \bm{v}\cdot \frac{\partial f}{\partial \bm{x}} + \frac{q[\bm{E}+\bm{v}\times \bm{B}]}{m}\cdot \frac{\partial f}{\partial \bm{v}} = \sum_j C(f_i, f_j) \label{eq:plasma4}
\end{eqnarray} 
By taking the second moment of Equation~\eqref{eq:plasma4} with respect to $\bm{v}$ (the particle's velocity), the following single-fluid equation can be used to describe the temperature evolution in magnetized plasma on macroscopic level.   
\begin{subequations}
	\begin{align}
	\frac{3}{2} n \frac{d T}{d t} &= -n T(\nabla \cdot \bm{u}) - \nabla \cdot \bm{q} - \bm{\pi} : \nabla \bm{u} + W \label{eq:plasma5} \\
	&= \frac{3}{2}\frac{\partial n}{\partial t}T + \frac{3}{2}\nabla\cdot(n\bm{u})T + \frac{3}{2} n \frac{\partial T}{\partial t} + \frac{3}{2} n \bm{u} \cdot \nabla T \label{eq:plasma5b}
	\end{align}
\end{subequations}  
where $n$ respectively are the number density of electrons (approximately the same to the number density of ions), the temperature and the fluid velocity. $d/dt$ in Equation~\eqref{eq:plasma5} refers to a sort of material derivative and is defined in Equation~\eqref{eq:plasma5b}. $\bm{\pi}: \nabla \bm{u}$ is the flow-stress-gradient and $W$ is the collisional energy exchange. For simplicity in the study of anisotropic heat conduction, the flow velocity $\bm{u}$ and collisional heating terms are ignored and a heat source term $Q$ is applied instead. Therefore, the resultant equation for unsteady anisotropic heat conduction becomes
\begin{eqnarray}
\frac{3}{2}n\frac{dT}{dt} = -\nabla \cdot \bm{q} + Q
\end{eqnarray}  
The kinetic heat flow vector $\bm{q}$ closure can be approximated by considering it, e.g,. for electrons, as the Fourier's law of heat conduction in Equation~\eqref{eq:plasma6}, where $\bm{\kappa}_s$ refers to the thermal Spitzer conductivity tensor. It means that the heat flow rate is proportional to the magnitude of the temperature's gradient, but opposite direction.
\begin{eqnarray}
\bm{q} = -\bm{\kappa}_s \nabla T \label{eq:plasma6}
\end{eqnarray}
Therefore, the general form for the heat conduction in plasma is
\begin{eqnarray}
\frac{3}{2} n \frac{dT}{dt} = \nabla \cdot (\bm{\kappa}_s \nabla T) + Q \label{eq:plasma7}
\end{eqnarray}

Due to the ionized nature of plasma and the presence of a strong magnetic field, the dynamics of particles of plasma and the associated energy dissipation is very non-isotropic. The charged particles, e.g., electrons and ions, move rapidly in tight spiral orbits, known as gyro-orbits, along the magnetic field lines, but tremendously slow along the normal direction, e.g., $\kappa_{\parallel} \gg \kappa_{\perp}$, typically 10 orders of magnitude difference. Because of anisotropic thermal conduction in the magnetized plasma, the thermal conduction tensor becomes a non-diagonal matrix and a sense of directionality is introduced into the normally uniformly dissipative characteristic of the diffusion term. Therefore, the temperature gradient is decomposed into three components/auxiliary vectors~\cite{goedbloed2004principles}, $\nabla_{\parallel} T$, $\nabla_{\perp} T$ and $\nabla_{\wedge} T$ defined in Equation~\eqref{eq:plasma8}, with respect to the unit direction of magnetic field $\bm{b} = \bm{B}/|\bm{B}|$, where $\nabla T = \nabla_{\parallel} T + \nabla_{\perp} T$.
$\nabla_{\parallel} T$ and $\nabla_{\perp} T$ are the auxiliary vectors along and normal to the magnetic field on the $\bm{b}-\nabla T$ plane respectively. Whereas, the auxiliary vector $\nabla_{\wedge} T$ accounts for the direction of electromagnetic induction and normal to the $\bm{b}-\nabla T$ plane.      

\begin{eqnarray}
\nabla_{\parallel} T = \bm{b}(\bm{b}\cdot \nabla T),\; \nabla_{\perp} T = (\bm{b}\times \nabla T)\times \bm{b}, \; \nabla_{\wedge} T = \bm{b} \times \nabla T \label{eq:plasma8}
\end{eqnarray}

It is shown by Goedbloed \& Poedts~\cite{goedbloed2004principles} that if a second rank tensor $\bm{\kappa}_s$ representing the anisotropic transport coefficients is symmetric with respect to rotations about the magnetic field $\bm{b}$, it implies that $\bm{\kappa}_s$ can only have three independent elements, e.g., $\kappa_{\parallel}$, $\kappa_{\perp}$ and $\kappa_{\wedge}$, and have the form below
\begin{eqnarray}
\bm{\kappa}_s = \begin{bmatrix}
\kappa_{\perp} & -\kappa_{\wedge} & 0 \\
\kappa_{\wedge} & \kappa_{\perp} & 0 \\
0 & 0 & \kappa_{\parallel}\\
\end{bmatrix}
\end{eqnarray} 
Assuming the magnetic field is parallel to the 3rd axis of $\nabla T$ and the temperature gradient can be expressed as $\nabla T = (\partial_1 T) \bm{\hat{e}}_1 + (\partial_2 T) \bm{\hat{e}}_2 + (\partial_{\parallel} T) \bm{\hat{e}}_{\parallel}$ and
\begin{eqnarray}
\nabla_{\parallel} T &=& (\partial_{\parallel} T) \bm{\hat{e}}_{\parallel}; \; \nabla_{\wedge} T = -(\partial_{2} T) \bm{\hat{e}}_{1} + (\partial_{1} T) \bm{\hat{e}}_{2}; \nonumber\\
 \nabla_{\perp} T &=& (\partial_1 T) \bm{\hat{e}}_1 + (\partial_2 T) \bm{\hat{e}}_2 
\end{eqnarray}
Subsequently the tensor and vector product can be written as
\begin{eqnarray}
\bm{\kappa}_s \cdot \nabla T &=& \begin{bmatrix}
\kappa_{\perp} & -\kappa_{\wedge} & 0 \\
\kappa_{\wedge} & \kappa_{\perp} & 0 \\
0 & 0 & \kappa_{\parallel}\\
\end{bmatrix} \cdot \begin{bmatrix}
\partial_1 T  \\ \partial_2 T \\ \partial_{\parallel} T
\end{bmatrix} \nonumber\\
&=& \big((\kappa_{\perp}\partial_1 T)-(\kappa_{\wedge}\partial_2 T)\big)\bm{\hat{e}}_1 + 
\big((\kappa_{\wedge}\partial_1 T)+(\kappa_{\perp}\partial_2 T)\big)\bm{\hat{e}}_2 \nonumber\\
& & (\kappa_{\parallel}\partial_{\parallel} T)\bm{\hat{e}}_{\parallel} \nonumber\\
&=& \kappa_{\parallel} \nabla_{\parallel} T + \kappa_{\wedge} \nabla_{\wedge} T + \kappa_{\perp} \nabla_{\perp} T \nonumber\\
&=& \kappa_{\parallel} \bm{b}(\bm{b}\cdot \nabla T) + \kappa_{\perp}\big(\nabla T - \bm{b}(\bm{b}\cdot \nabla T)\big) + \kappa_{\wedge}\bm{b}\times\nabla T
\end{eqnarray}
Therefore the general form of anisotropic thermal conduction of magnetized plasma in Equation~\eqref{eq:plasma7} can be recast into the form below
\begin{eqnarray}
\frac{3}{2} n \frac{dT}{dt} &=& \nabla \cdot \Big[\kappa_{\parallel} \bm{b}(\bm{b}\cdot \nabla T) + \kappa_{\perp}\big(\nabla T - \bm{b}(\bm{b}\cdot \nabla T)\big) + \kappa_{\wedge}\bm{b}\times\nabla T\Big] + Q \nonumber\\
&=& \nabla \cdot \Big[\kappa_{\parallel} (\bm{b}\otimes \bm{b}) \cdot \nabla T + \kappa_{\perp}(\bm{I}-\bm{b}\otimes\bm{b})\cdot\nabla T + \kappa_{\wedge}\bm{b}\times\nabla T\Big] + Q \nonumber\\
&=& \nabla \cdot \Big[\big((\kappa_{\parallel}-\kappa_{\perp}) (\bm{b}\otimes \bm{b})  + \kappa_{\perp}\bm{I}\big)\cdot\nabla T\Big] + \nabla \cdot [\kappa_{\wedge}\bm{b}\times\nabla T] + Q \nonumber\\
&=& \nabla \cdot \big(\bm{\kappa}_c \cdot \nabla T \big) + \nabla \cdot [\kappa_{\wedge}\bm{b}\times\nabla T] + Q \label{eq:plasma9}
\end{eqnarray}
where $\bm{I}$ is an identity matrix. $\bm{\kappa}_c$ is defined as the thermal conductivity tensor and shown below
\begin{eqnarray}
\bm{\kappa}_c &=& (\kappa_{\parallel}-\kappa_{\perp}) (\bm{b}\otimes \bm{b})  + \kappa_{\perp}\bm{I} \nonumber\\
&=& (\kappa_{\parallel}-\kappa_{\perp}) \begin{bmatrix}
b^2_x & b_x b_y \\ b_x b_y & b^2_y 
\end{bmatrix} + \begin{bmatrix}
\kappa_{\perp} & 0\\ 0 &  \kappa_{\perp}
\end{bmatrix} \nonumber\\
&=& \begin{bmatrix}
(\kappa_{\parallel}-\kappa_{\perp}) b^2_x + \kappa_{\perp} & (\kappa_{\parallel}-\kappa_{\perp}) b_x b_y\\
(\kappa_{\parallel}-\kappa_{\perp}) b_x b_y & (\kappa_{\parallel}-\kappa_{\perp}) b^2_y + \kappa_{\perp} 
\end{bmatrix}
\end{eqnarray}
If there is an angle $\theta$ defining the direction of the 2D magnetic field, $\bm{b}$ can be defined as $\bm{b} = [b_x, b_y]' = [cos(\theta), sin(\theta)]' = [cs, ss]'$, where the superscript ($'$) is a transpose operator. Consequently, $b^2_x = cs^2$, $b_x b_y = cs \; ss$ and $b^2_y = ss^2$.

In this study, since it is assumed that the anisotropic thermal conduction in magnetized plasma is two-dimensional, the induction direction $\kappa_{\wedge}$ in the 3rd dimension (the term $\nabla \cdot [\kappa_{\wedge}\bm{b}\times\nabla T]$ in Equation~\eqref{eq:plasma9}) is neglected. Therefore, the implemented strong form of two-dimensional anisotropic thermal conduction becomes

\begin{eqnarray}
\frac{3}{2} n \frac{dT}{dt} = \nabla \cdot \begin{bmatrix}
\big((\kappa_{\parallel}-\kappa_{\perp}) cs^2 + \kappa_{\perp}\big)\partial_x T + \big((\kappa_{\parallel}-\kappa_{\perp}) cs\; ss\big)\partial_y T\\
\big((\kappa_{\parallel}-\kappa_{\perp}) cs\; ss\big)\partial_x T + \big((\kappa_{\parallel}-\kappa_{\perp}) ss^2 + \kappa_{\perp})\partial_y T 
\end{bmatrix} + Q \label{eq:plasma10}
\end{eqnarray} 

\section{Key parameters}
Braginskii's transport coefficients are widely used in tokamak edge modelling. The Barginskii transport coefficients~\cite{braginskii1965transport,morsenuclear,richardson20192019} for ions (i) and electrons (e) are defined in Equation~\eqref{eq:plasma14}.
\begin{subequations} \label{eq:plasma14}
	\begin{align}
	\kappa^e_{\parallel} &= 3.2\frac{n k_B T_e \tau_e}{m_e}; \; \kappa^e_{\perp} = 4.7 \frac{n k_B T_e}{m_e \omega^2_{ce} \tau_e};\; \kappa^e_{\wedge} = \frac{5}{2}\frac{n k_B T_e }{m_e \omega_{ce}}\\
	\kappa^i_{\parallel} &= 3.9\frac{n k_B T_i \tau_i}{m_i}; \; \kappa^i_{\perp} = 2.0 \frac{n k_B T_i}{m_i \omega^2_{ci} \tau_i};\; \kappa^i_{\wedge} = \frac{5}{2}\frac{n k_B T_i}{m_i \omega_{ci}}
	\end{align}
\end{subequations}  
where $T_{\alpha}$, $\omega_{c\alpha}$, $\tau_{\alpha}$ and $m_{\alpha}$ respectively are the temperature, the cyclotron frequency (or gyro-frequency), the collision time and the mass of electrons ($\alpha=e$) and ions ($\alpha=i$). $k_B T$ [K] = $|e|T$ [eV], where $k_B$ is the Boltzmann's constant and $|e|$ is the absolute value of the charge on the electron. 1.0 [eV] $\approx 1.6 \times 10^{-19}$ [J] $\approx 1.16 \times 10^{4}$ [K]. The cyclotron frequency of the electron and the ion are defined as 
\begin{eqnarray} \label{eq:plasma15}
\omega_{c e} = \frac{e B}{m_e}\; \omega_{ci} = \frac{e B Z}{m_i} = \frac{e B Z}{m_p A}\label{eq:freq}
\end{eqnarray}
where $e$, $m_p$ and $B$ are the positive elementary charge, the mass of proton and the magnitude of magnetic field. $A = m_i/m_p$. $Z$ is the ion charge state. The above definitions of cyclotron frequency is in SI unit. In Gaussian units, the Lorentz force in Equation~\eqref{eq:freq} differs by a factor of $1/c$, where $c$ is the speed of light. The collision time of the electrons and ions are defined as
\begin{subequations} \label{eq:plasma16}
	\begin{align}
	\tau_e &= \frac{3\sqrt{m_e}(k_B T_e)^{3/2}}{4 n \sqrt{2\pi} \lambda e^4} = 6\sqrt{2\pi^3} \frac{\epsilon^2_0 \sqrt{m_e}}{e^4} \frac{(k_B T_e)^{3/2}}{Z^2 n \lambda} \\
	\tau_i &= \frac{3\sqrt{m_i}(k_B T_i)^{3/2}}{4 n \sqrt{\pi}\lambda e^4} = 12\sqrt{\pi^3} \frac{\epsilon^2_0 \sqrt{m_p}}{e^4} \frac{(k_B T_i)^{3/2} \sqrt{A}}{Z^4 n \lambda}
	\end{align}
\end{subequations}
where $\lambda = ln(\Lambda)$ is the Coulomb logarithm. Noted that $\lambda$ depends on the type of collisions, and is typically between 10 and 20 in a fusion plasma~\cite{spatschek2013high}. $\epsilon_0$ is the permittivity of free space.

Due to the anisotropic nature of the magnetized plasma, the magnitudes of diffusion coefficients for ions and electrons are very disparate, so one or other might be neglected~\cite{arter2020}. Assuming $B$ is of order unity (in Tesla) and $T_e \approx T_i$, so $\kappa_{\parallel} \gg \kappa_{\perp}$. Hence the thermal conductivity coefficients in Equation~\eqref{eq:plasma10} are chosen as $\kappa_{\parallel} \approx \kappa^e_{\parallel}$ and $\kappa_{\perp} \approx \kappa^i_{\perp}$. Since the two-dimensional anisotropic thermal conduction is considered in this study, $\kappa^e_{\wedge}=\kappa^i_{\wedge}=0$. By substituting Equation~\eqref{eq:freq} and Equation~\eqref{eq:plasma16} into $\kappa^e_{\parallel}$ and $\kappa^i_{\perp}$ in Equation~\eqref{eq:plasma14}, the $\kappa_{\parallel}$ and $\kappa_{\perp}$ can be written as

\begin{eqnarray}
\kappa_{\parallel} = 19.2 \sqrt{2\pi^3} \frac{1}{\sqrt{m_e}} \frac{\epsilon^2_0}{e^4} \frac{(k_B T_e)^{5/2}}{Z^2 \lambda} \quad \kappa_{\perp} = \frac{1}{6\sqrt{\pi^3}} \frac{1}{m_i} \Big(\frac{n Z e}{B \epsilon_0}\Big)^2 \frac{(m_p A)^{3/2}\lambda}{\sqrt{k_B T_i}}
\end{eqnarray} 


\section{Variational formulation}
The high-order spectral/hp element method~\cite{karniadakis2013spectral} is employed to numerically solve the anisotropic thermal conduction equations in Section~\ref{sec:geq}. At first, Equation~\eqref{eq:plasma10} is spatially discretized by spectral element method into a semi-discrete formulation. Subsequently, a suitable time integration scheme is applied to march the approximated solution in time space. We first define an appropriate set of finite trial function space ($S^h$) for the temperature field and its corresponding finite test function spaces ($V^h$), where the superscript ($h$) denotes a finite function space, e.g, $S^h \subset S$. Hence the variational formulation of Equation~\eqref{eq:plasma10} can be written as follow: for all $\psi^h \in V^h$, find $T^h \in S^h$ such that Equation~\eqref{eq:plasma11} is satisfied. 
\begin{eqnarray}
& &\int_{\Omega} \psi^h \Big[\frac{3}{2} n \frac{d T^h}{d t} -\nabla \cdot (\bm{\kappa}_c \cdot \nabla T^h) -Q^h\Big] d\Omega = 0 \quad \forall \psi^h \in V^h \nonumber\\
&\implies&\frac{3}{2} n \int_{\Omega} [\psi^h\frac{d T^h}{d t}]d\Omega - \int_{\Omega} [\psi^h \nabla \cdot (\bm{\kappa}_c \cdot \nabla T^h)] d \Omega  = \int_{\Omega} [\psi^h Q^h] d\Omega \label{eq:plasma11}
\end{eqnarray}
By applying the divergence theorem to the second term in Equation~\eqref{eq:plasma11}, which has the second derivative, the secondary variable of the problem can be obtain in Equation~\eqref{eq:plasma12} to account for the natural boundary condition, e.g., the heat flux.
\begin{eqnarray}
-\int_{\Omega} [\psi^h \nabla \cdot (\bm{\kappa}_c \cdot \nabla T^h)]d\Omega &=& -\int_{\Gamma} [\psi^h (\bm{\kappa}_c \cdot \nabla T^h)\cdot \bm{n}] d\Gamma \nonumber\\
& &+ \int_{\Omega} [\nabla \psi^h \cdot (\bm{\kappa}_c \cdot \nabla T^h)]d\Omega \label{eq:plasma12}
\end{eqnarray}
where $\Omega$ and $\Gamma$ refers to the geometric domain and boundary respectively. $\bm{n}$ denotes the outward normal vector. Consequently, the variational formulation of two-dimensional anisotropic thermal conduction of magnetized plasma can be written as

\begin{eqnarray}
\frac{3}{2} n \int_{\Omega} [\psi^h\frac{d T^h}{d t}]d\Omega &+& \int_{\Omega} [\nabla \psi^h \cdot (\bm{\kappa}_c \cdot \nabla T^h)]d\Omega  = \int_{\Omega} [\psi^h Q^h]d\Omega \nonumber\\
& &+ \int_{\Gamma} [\psi^h (\bm{\kappa}_c \cdot \nabla T^h)\cdot \bm{n}] d\Gamma  \quad \forall \psi^h \in V^h  \label{eq:plasma13}
\end{eqnarray} 
Assuming a steady case without external heat source, the first and third terms in Equation~\eqref{eq:plasma13} can be neglected. On the other hand, an appropriate time integration scheme should be employed for an unsteady simulation of the anisotropic thermal conduction. 









\nocite{*}
\bibliographystyle{unsrt}
\bibliography{references}


\end{document}
